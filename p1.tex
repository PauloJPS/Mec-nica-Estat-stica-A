\documentclass[12pt]{article}
\usepackage[a4paper]{geometry}
\usepackage[myheadings]{fullpage}
\usepackage{fancyhdr}
\usepackage{lastpage}
\usepackage{graphicx, wrapfig,  setspace, booktabs}
\usepackage[T1]{fontenc}
\usepackage[font=small, labelfont=bf]{caption}
\usepackage{fourier}
\usepackage[protrusion=true, expansion=true]{microtype}
\usepackage{sectsty}
\usepackage{url, lipsum}
\usepackage[utf8]{inputenc}
\usepackage[T1]{fontenc}
\usepackage[portuguese]{babel}
\usepackage{amsmath}
\usepackage{subcaption}
\usepackage{graphicx}
\usepackage{listings}





\newcommand{\HRule}[1]{\rule{\linewidth}{#1}}
\onehalfspacing
\setcounter{tocdepth}{5}
\setcounter{secnumdepth}{5}

%-------------------------------------------------------------------------------
% HEADER & FOOTER
%-------------------------------------------------------------------------------

%-------------------------------------------------------------------------------
% TITLE PAGE
%-------------------------------------------------------------------------------

\begin{document}

\title{ \normalsize \textsc{Paulo José Paulino de Souz}
		\\ [2.0cm]
		\HRule{0.5pt} \\
		\LARGE \textbf{\uppercase{Prova 1 de Mecânica Estatística A}}
		\HRule{2pt} \\ [0.5cm]
		\normalsize  \vspace*{5\baselineskip}}



\author{Paulo José - 9283890}
\maketitle
\newpage

%-------------------------------------------------------------------------------
% Section title formatting
\sectionfont{\scshape}
\section*{Exercício 1}
A função característica é definida como uma transformada de Fourier da função densidade de probabilidade, sendo assim, para uma função característica aplicando uma transformada inversa de Fourier obtemos a densidade de probabilidade, 
\begin{equation}
f(x) = \frac{1}{2\pi}\int\limits_{-\infty}^{\infty}e^{-ikx}f(k)dk \quad. 
\end{equation}
Usando a função do enunciado, 
\begin{equation}
g(k) = (a + bcos(k)),
\end{equation}
a transformada inversa,
\begin{equation}
\begin{split}
f(x) = \frac{1}{2\pi}\int\limits_{-\infty}^{\infty}(a + bcos(k))dk = \\
\frac{1}{2\pi}\int\limits_{-\infty}^{\infty}e^{-ikx}a dk + \frac{b}{4\pi}\int\limits_{-\infty}^{\infty} e^{-ikx}\left(e^{ik} + e^{-ik}\right)dk = \\
a\delta(x) + \frac{b}{4\pi}\left( \int\limits_{-\infty}^{\infty}e^{-ik(x - 1)}dk + \int\limits_{-\infty}^{\infty}e^{-ik(x + 1)}dk \right)
\end{split}\quad,
\end{equation}
toma, então a forma:
\begin{equation}
f(x) = a\delta(x) + \frac{b}{2}\left(\delta(x-1) + \delta(x+1)\right) \quad.
\end{equation}
Comparando essa expressão final com a equação que leva uma distribuição discreta para uma densidade,
\begin{equation}
f(x) = \sum_l p_l\delta(x - x_l) \quad ,
\end{equation}
obtêm-se a forma discreta da função de probabilidade:
\begin{equation}
\begin{split}
&p_l = a,\quad l = 0 \quad,\\
&p_l = b/2,\quad l = 1 \quad,\\
&p_l = b/2,\quad l = -1 \quad. 
\end{split}
\end{equation}

Os momentos são calculados usando a função característica pela fórmula:
\begin{equation}
\mu_n = i^n \frac{dg(k)}{dk}\Big\rvert_{k=0} \quad.
\end{equation}
Calculando os primeiros momentos,
\begin{equation}
\begin{split}
\frac{d (a + b\cos(k))}{dk} = -b\sin(k) \quad,
\mu_1 = -b\sin(k)\rvert_{k=0} = 0.\\ 
\frac{d (-b\sin(k))}{dk} = -b\cos(k) \quad, \mu_2 = -(i)^2b\cos(k)\rvert_{k=0} = b. 
\end{split}
\end{equation}
Nota-se que existe uma periodicidade nos momentos, os ímpares serão sempre \textbf{zero} e os pares serão sempre $b$.   Sobre os momentos pares, temos a recorrência,
\begin{equation}
\mu_{n=par} = (i)^n b\frac{d^n\cos(k)}{dk} = (i)^n(-1)^{n+1} b\cos(k)\rvert_{k=0} = b.  
\end{equation}


Uma pequena observação a ser feita é que foi usado a definição da delta pela normalização de Fourier,
\begin{equation}
\delta(x-a) = \frac{1}{2\pi}\int\limits_{-\infty}^{\infty} e^{-ik(x-a)}dk \quad.
\end{equation}

\section*{Exercício 2}
A função de probabilidade de um processo de Poisson é discreta e dada pela lei,
\begin{equation}
p_l = \frac{e^{-\lambda} \lambda^l}{l!}, \quad l=\in  \mathcal{N}.
\end{equation}
O objetivo aqui é encontrar uma expressão para os cumulantes e para isso irei recorrer para a função característica. Antes de usar a transformada de Fourier é necessário escrever $p_l$ como uma densidade de probabilidade, 
\begin{equation}
\begin{split}
f(x) = \sum_l p_l\delta(x-l) \quad,\\
f(x) = \sum_l \frac{e^{-\lambda} \lambda^l\delta(x-l)}{l!} \quad.
\end{split}
\end{equation}
Assim a função característica vem naturalmente, 
\begin{equation}
\begin{split}
g(k) = \int e^{ikx}\sum_l \frac{e^{-\lambda} \lambda^l\delta(x-l)}{l!} dx\\
g(k) = \sum_l \frac{e^{-\lambda} \lambda^l e^{ikl}}{l!} \quad.
\end{split}
\end{equation}
A exponencial em $\lambda$ sai para fora da soma e fatorando o expoente $l$, obtemos a expressão:
\begin{equation}
g(k) = e^{-\lambda}\sum_l \frac{(\lambda^l e^{ik})^l}{l!} = e^{-\lambda}e^{\lambda e^{ik}} = e^{\lambda(e^{ik} - 1)} \quad. 
\end{equation}
A função geradora de cumulantes é a expansão do $\ln$ da função característica,
\begin{equation}
\ln g(k) = \sum\limits{n=1} \frac{(ik)^n \kappa_n}{n!},
\end{equation}
e aplicando para a função característica encontrada,
\begin{equation}
\ln g(k) = \lambda(e^{ik} - 1) = \lambda\sum_{n=1} \frac{1}{n!}\frac{d^n e^{ik}}{dk}\Big\rvert_{k=0} = \lambda\sum_{n=1} \frac{(ik)^n}{n!} \quad.
\end{equation}
Comparado a expansão da função característica com a função geradora de cumulantes concluí-se que todos os cumulantes são iguais e assumem o valor de $\lambda$.
\begin{equation}
\kappa_n = \lambda, \quad n=1, 2, 3, \cdots
\end{equation}

\section*{Exercício 3}
O nosso problema se resume em um \textit{bêbado} unidimensional que anda para direita com probabilidade $q/2$, para esquerda com probabilidade $q/2$ e, como está muito bêbado, fica parada com probabilidade $p$. Pela normalização, $p+q=1$. A função de probabilidade desse problema é então,
\begin{equation}
\begin{split}
p_l = p, \quad l=0, \\
p_l = \frac{q}{2}, \quad l=0,\\
p_l = \frac{q}{2}, \quad l=0.
\end{split}
\end{equation}
No contínuo a densidade de probabilidade é dada pela fórmula,
\begin{equation}
f(x) = \sum_l p_l\delta(x-l) \quad,
\end{equation}
e combinada com a definição de função característica, obtemos:
\begin{equation}
g(k) = \sum_l p_l e^{ikl} = pe^{-ik0} + \frac{q}{2}e^{ik} + \frac{q}{2}e^{-ik} \quad, 
\end{equation}
Prosseguindo, defina a variável aleatória que é a soma de variáveis,
\begin{equation}
m = \sum_i \sigma_i \quad,
\end{equation}
onde $\sigma_i$ são os ensaios, passos, do \textit{bêbado}. Temos que a função característica de uma soma de variáveis aleatórias uniformes e identicamente distribuídas é dada por:
\begin{equation}
g_m(k) = \prod\limits_{i=1}^N g(k) = g(k)^N \quad,  
\end{equation}
que nesse caso assume a forma:
\begin{equation}
g_m(k) = (p + \frac{q}{2}e^{ik} + \frac{q}{2}^e{-ik})^N,
\end{equation}
que expandindo em trinômio de Newton, obtemos:
\begin{equation}
g_m(k) = \sum\limits_{l=0}^N \sum\limits_{\omega=0}^l \frac{N!}{(N-l)!(l-\omega)!\omega !} p^{N-l} \left(\frac{q}{2}e^{ik}\right)^{l-\omega}\left(\frac{q}{2}e^{-ik}\right)^{\omega} \quad,
\end{equation}
organizando os termos de forma que haja apenas uma exponencial para que seja aplicada a função inversa de fourier,
\begin{equation}
g_m(k) = \sum\limits_{l=0}^N \sum\limits_{\omega=0}^l \frac{N!}{(N-l)!(l-\omega)!\omega !} p^{N-l} \left(\frac{q}{2}\right)^{l}e^{ik(l-2\omega)} \quad.
\end{equation}
Agora é necessário fazer uma manipulação nas variáveis do somatório para que ela concorde com o argumento da exponencial. É natural escolher a nova variável como a do argumento da exponencial, assim defina:
\begin{equation}
m = l - 2\omega \quad,
\end{equation}
e com isso troca-se a variável do primeiro somatório por $m$. Com isso deve-se analisar os limites da nova variável. O valor máxima que $m$ pode tomar é quando $\omega=0$ e quando $l=N$ e assim o limite superior de $m$ é $N$. O limite inferior é quando $l=N$ e quando $\omega = l = N$, então $m=-N$. A variação de $m$ não é unitária, pois para um $l$ fixo, por exemplo $l=10$ e $\omega=0$, o próximo passo é $l=10$ e $\omega=1$, então, $m$ foi de $10$ para $10-2=8$. Com isso se conclui que $m$ varia de dois em dois. O somatório em $\omega$ também sobre modificações e para avaliar considere o expoente de $p$. Esse expoente deve sempre estar entre os limites:
\begin{equation}
0 \leq N-m-2\omega \leq N \quad,
\end{equation}
dessa forma se $m=-N$ então $\omega$ deve ser igual a zero e caso $m=N$ então $\omega$ deve ser igual a $N$.

Reescrevendo os somatórios substituindo $l=m+2\omega$,
\begin{equation}
g_m(k) = \sum\limits_{m=-N}^N \sum\limits_{\omega=0}^N \frac{N!}{(N-m-2\omega)!(m + \omega)!\omega !} p^{N-m-2\omega} \left(\frac{q}{2}\right)^{m+\omega}e^{ikm} \quad,
\end{equation}
conseguimos calcular a distribuição de probabilidades, 
\begin{equation}
P_N(m) = \sum\limits_{\omega=0}^N \frac{N!}{(N-m-2\omega)!(m + \omega)!\omega !} p^{N-m-2\omega} \left(\frac{q}{2}\right)^{m+\omega}.
\end{equation}
Esse resultado nos fala e seguinte, para cada distância $m$ percorrida temos que levar em conta as possibilidade tanto do bêbado ficar parado quando as possibilidades do bêbado ir e voltar. 

Como as variáveis aleatórias são uniformes e identicamente distribuídas com os momentos bem definidos podemos utilizar a lei dos grandes números para encontrar a distribuição de probabilidade para $N > 1$. Sabemos que a distribuição nesse limite deve ter a forma,
\begin{equation}
P_m(N) = \frac{1}{\sqrt{2\pi \kappa_2}}\exp{\left[-\frac{(m-\kappa_1)^2}{2\kappa_2}\right]} \quad ,
\end{equation}
onde $\kappa_1$ e $\kappa_2$ são o primeiro e segundo cumulantes e eles se relacional com o primeiro segundo momento da seguinte forma,
\begin{equation}
\kappa_1 = \mu_1, \quad \kappa_2 = \mu_2 - \mu_1^2\quad. 
\end{equation}
Como os momentos são mais fáceis de calcular vou calcula-los a partir da função característica. O primeiro momento:
\begin{equation}
\begin{split}
\frac{d(p + q\cos(k))^N}{dk} = N(p + q\cos(k))^{N-1}(-q\sin(k))\\
\mu_1 = i\frac{idg(k)}{dk}\Big\rvert_{k=0} = 0. 
\end{split}
\end{equation}
Aqui foi usado a definição dos cossenos por meio de exponenciais complexas. O segundo momento é então calculado:
\begin{equation}
\begin{split}
i^2\frac{d^2g(k)}{dk^2} =  N(N-1)(p + q\cos(k))^{N-2}(-q\sin(k))^2 +\\
N(p + q\cos(k)^{N-1}(-q\cos(k))\\
\mu_2 = Nq,
\end{split}
\end{equation}
onde o cosseno vai a um e $p+q=1$. 
Com isso escrevemos a distribuição de probabilidade,
\begin{equation}
P_m(N) = \frac{1}{\sqrt{2\pi Nq}}\exp{\left[-\frac{m^2}{2Nq}\right]}
\end{equation}

\section*{Exercício 4}
Neste problema temos um \textit{bêbado} unidimensional que está muito, mais muito bêbado, que ficou coxo. Devido a esse problema seus passos são dados com funções de probabilidade diferentes dependendo se ele está no passo par ou no ímpar. Para o passo par ele tem a função de probabilidade,
\begin{equation}
\begin{split}
p_l = p, \quad l=1,\\
p_l = q, \quad l=-1,
\end{split}
\end{equation} 
e para o passo ímpar ele tem a função de probabilidade dada por:
\begin{equation}
\begin{split}
p_l = q, \quad l=1,\\
p_l = p, \quad l=-1,
\end{split}
\end{equation}
com função características,
\begin{equation}
\begin{split}
g_p(k) = (p^{ik} + qe^{-ik}) \quad, \\
g_p(k) = (q^{ik} + pe^{-ik}) \quad.
\end{split}
\end{equation}
Para encontrar a distância percorrida por esse bêbado considere a variável aleatória,
\begin{equation}
m = \sum_i \sigma_i \quad,
\end{equation}
onde o $\sigma_i$ é ensaio, passo, do bêbado no tempo $i$. Como temos que os valores de $sigma_i$ mudam conforme a paridade do intervalo de tempo, é natural separar a soma em duas, uma que percorre sobre intervalos pares e a outra sobre intervalos ímpares,
\begin{equation}
m = \sum_{i, par} \sigma_i + \sum_{j, ímpar}\sigma_j\quad. 
\end{equation}
A função característica de $m$ deverá ser dividida em dois casos um caso com o número de passos fixo em um número par e o outro caso num número ímpar. Isso ficará mais claro no desenvolvimento da matemática. Considere a função característica de $m$ para um caso par,
\begin{equation}
g_{N,par}(k) = \prod\limits_{i=0}^N g(k) = g_p(k)^{N/2}g_i(k)^{N/2}, 
\end{equation}
onde foram dados $N/2$ passos em intervalos pares e $N/2$ em intervalos ímpares. Como os expoentes são iguais podemos multiplica-los,
\begin{equation}
g_{N,par}(k) = (p^2 + q^2 + pq(e^{ik} + e^{-ik})^{N/2} = (p^2 + q^2 + 2pq\cos(k))^{N/2} \quad.
\end{equation}
A semelhança com o problema $3$ nos leva interpretação que este problema é análogo ao problema de um bêbado que ficar parado com probabilidade $p^2 + q^2$ e anda pra direita ou esquerda com probabilidade $pq$. Continuando, como o interesse é encontrar a distribuição para $N$ grande vale usar do teorema do limite central e escrever a distribuição de probabilidades da forma,
\begin{equation}
P_{N}(m) = \frac{1}{\sqrt{2\pi\kappa_2}}\exp{\left(-\frac{(m-\kappa_1)^2}{2\kappa_2}\right)}, 
\end{equation}
então, calculando os momentos,
\begin{equation}
\begin{split}
\frac{d(p2 + q^2 + 2qp\cos(2k))^{N/2}}{dk} = \frac{N}{2}\left(p2 + q^2 + 2qp\cos(2k)\right)^{N/2 - 1}\left(-4pq\sin(2k)\right)\\
\mu_1 = i\frac{idg(k)}{dk}\Big\rvert_{k=0} = 0. 
\end{split}
\end{equation}
\begin{equation}
\begin{split}
i^2\frac{d^2g(k)}{dk^2} =  \frac{N}{2}\left(\frac{N}{2}-1)(p^2 + q^2 + 2pq\cos(2k)\right)^{\frac{N}{2}-2}\left(-4p^2q^2\sin(2k)\right)^2 +\\
+ \left(p^2 + q^2 + 2pq\cos(2k)\right)^{\frac{N}{2}}(-4pq\cos(2k))\\
\mu_2 = 4Npq.
\end{split}
\end{equation}
Com isso escreve-se a distribuição de probabilidade para o caso par,
\begin{equation}
P_{N, par}(m) = \frac{1}{\sqrt{4\pi Npq}}\exp{\left(-\frac{(m)^2}{4Npq}\right)}.
\end{equation}
Para $N$ ímpar a função característica possui termo a mais, pois tem um intervalo de tempo ímpar que não pode ser dividido no $N/2$, 
\begin{equation}
g_{N,ímpar}(k) = (qe^{ik} + pe^{-ik})(p^2 + q^2 + 2pq\cos(k))^{N/2} \quad.
\end{equation}
O cálculo do primeiro momento,
\begin{equation}
\begin{split}
&\frac{d(p2 + q^2 + 2qp\cos(2k))^{N/2}(qe^{ik} + pe^{-ik})}{dk} = \\
&= \frac{N}{2}\left(p2 + q^2 + 2qp\cos(2k)\right)^{N/2 - 1}\left(-4pq\sin(2k)\right)(qe^{ik} + pe^{-ik})- \\
& - (p^2 + q^2 + 2qp\cos(2k))^{N/2}(iqe^{ik}-ipe^{-ik})\\
&\mu_1 = i\frac{idg(k)}{dk}\Big\rvert_{k=0} = i*(iq - ip) = p - q. 
\end{split}
\end{equation}
O segundo momento é trabalhoso e por isso, observe os termos que serão derivados. Na primeira parcela o primeiro termo sera derivado e aparecerá outro seno que vai zerar tudo, o segundo termo vai aparecer um cosseno e o terceiro o seno mata. Na segunda parcela a derivada do primeiro termo gera o seno que mata a parcela e a derivada do segundo termo muda o sinal. Dessa forma,
\begin{equation}
\begin{split}
\frac{d^2g_{impar}(k)}{dk^2} = \frac{N}{2}\left(p2 + q^2 + 2qp\cos(2k)\right)^{N/2 - 1}\left(-8pq\cos(2k)\right)(qe^{ik} + pe^{-ik}) + \\
+ (p^2 + q^2 + 2qp\cos(2k))^{N/2}(iqe^{ik}-ipe^{-ik})
\\
\mu_2 = \frac{i^2d^2g(k)}{dk}\Big\rvert_{k=0} = 4Npq - (p -q). 
\end{split}
\end{equation}
Com isso calcula-se o segundo cumulante,
\begin{equation}
\kappa_2 = \mu_2 - \mu_1 = 4Npq - q + p - (p - q)^2 = 4Npq - q + p -(p - q)^2 \quad, 
\end{equation}
Mas como $N$ é grande os termos de $p$ e $q$ sozinhos são desprezíveis, então, para o caso de número de passos ímpares temos:
\begin{equation}
P_{N, par}(m) = \frac{1}{\sqrt{4\pi Npq}}\exp{\left(-\frac{(m-(p-q))^2}{4Npq}\right)}.
\end{equation}  

\section*{Exercício 7}
A equação de Fokker-Planck pode ser escrita como uma equação de continuidade,
\begin{equation}
\frac{\partial P(x,t)}{\partial t} = \sum_i\left( \frac{\partial J(x, t)}{\partial t}\right),
\end{equation}
onde $J$ é a corrente de probabilidade. 

A entropia é definida como:
\begin{equation}
S(t) = -\int P(x,t) \ln P(x,t) dx
\end{equation}
\end{document}