\documentclass[12pt]{article}
\usepackage[a4paper]{geometry}
\usepackage[myheadings]{fullpage}
\usepackage{fancyhdr}
\usepackage{lastpage}
\usepackage{graphicx, wrapfig,  setspace, booktabs}
\usepackage[T1]{fontenc}
\usepackage[font=small, labelfont=bf]{caption}
\usepackage{fourier}
\usepackage[protrusion=true, expansion=true]{microtype}
\usepackage{sectsty}
\usepackage{url, lipsum}
\usepackage[utf8]{inputenc}
\usepackage[T1]{fontenc}
\usepackage[portuguese]{babel}
\usepackage{amsmath}
\usepackage{subcaption}
\usepackage{graphicx}
\usepackage{listings}





\newcommand{\HRule}[1]{\rule{\linewidth}{#1}}
\onehalfspacing
\setcounter{tocdepth}{5}
\setcounter{secnumdepth}{5}

%-------------------------------------------------------------------------------
% HEADER & FOOTER
%-------------------------------------------------------------------------------

%-------------------------------------------------------------------------------
% TITLE PAGE
%-------------------------------------------------------------------------------

\begin{document}

\title{ \normalsize \textsc{Paulo José Paulino de Souz}
		\\ [2.0cm]
		\HRule{0.5pt} \\
		\LARGE \textbf{\uppercase{Prova 1 de Mecânica Estatística A}}
		\HRule{2pt} \\ [0.5cm]
		\normalsize  \vspace*{5\baselineskip}}



\author{Paulo José - 9283890}
\maketitle
\newpage

%-------------------------------------------------------------------------------
% Section title formatting
\sectionfont{\scshape}
\section*{Exercício 1}
A função característica é definida como uma transformada de Fourier da função densidade de probabilidade, sendo assim, para uma função característica aplicando uma transformada inversa de Fourier obtemos a densidade de probabilidade, 
\begin{equation}
f(x) = \frac{1}{2\pi}\int\limits_{-\infty}^{\infty}e^{-ikx}f(k)dk \quad. 
\end{equation}
Usando a função do enunciado, 
\begin{equation}
g(k) = (a + bcos(k)),
\end{equation}
a transformada inversa,
\begin{equation}
\begin{split}
f(x) = \frac{1}{2\pi}\int\limits_{-\infty}^{\infty}(a + bcos(k))dk = \\
\frac{1}{2\pi}\int\limits_{-\infty}^{\infty}e^{-ikx}a dk + \frac{b}{4\pi}\int\limits_{-\infty}^{\infty} e^{-ikx}\left(e^{ik} + e^{-ik}\right)dk = \\
a\delta(x) + \frac{b}{4\pi}\left( \int\limits_{-\infty}^{\infty}e^{-ik(x - 1)}dk + \int\limits_{-\infty}^{\infty}e^{-ik(x + 1)}dk \right)
\end{split}\quad,
\end{equation}
toma, então a forma:
\begin{equation}
f(x) = a\delta(x) + \frac{b}{2}\left(\delta(x-1) + \delta(x+1)\right) \quad.
\end{equation}
Comparando essa expressão final com a equação que leva uma distribuição discreta para uma densidade,
\begin{equation}
f(x) = \sum_l p_l\delta(x - x_l) \quad ,
\end{equation}
obtêm-se a forma discreta da função de probabilidade:
\begin{equation}
\begin{split}
&p_l = a,\quad l = 0 \quad,\\
&p_l = b/2,\quad l = 1 \quad,\\
&p_l = b/2,\quad l = -1 \quad. 
\end{split}
\end{equation}

Os momentos são calculados usando a função característica pela fórmula:
\begin{equation}
\mu_n = i^n \frac{dg(k)}{dk}\Big\rvert_{k=0} \quad.
\end{equation}
Calculando os primeiros momentos,
\begin{equation}
\begin{split}
\frac{d (a + b\cos(k))}{dk} = -b\sin(k) \quad,
\mu_1 = -b\sin(k)\rvert_{k=0} = 0.\\ 
\frac{d (-b\sin(k))}{dk} = -b\cos(k) \quad, \mu_2 = -(i)^2b\cos(k)\rvert_{k=0} = b. 
\end{split}
\end{equation}
Nota-se que existe uma periodicidade nos momentos, os ímpares serão sempre \textbf{zero} e os pares serão sempre $b$.   Sobre os momentos pares, temos a recorrência,
\begin{equation}
\mu_{n=par} = (i)^n b\frac{d^n\cos(k)}{dk} = (i)^n(-1)^{n+1} b\cos(k)\rvert_{k=0} = b.  
\end{equation}


Uma pequena observação a ser feita é que foi usado a definição da delta pela normalização de Fourier,
\begin{equation}
\delta(x-a) = \frac{1}{2\pi}\int\limits_{-\infty}^{\infty} e^{-ik(x-a)}dk \quad.
\end{equation}

\section*{Exercício 2}
A função de probabilidade de um processo de Poisson é discreta e dada pela lei,
\begin{equation}
p_l = \frac{e^{-\lambda} \lambda^l}{l!}, \quad l=\in  \mathcal{N}.
\end{equation}
O objetivo aqui é encontrar uma expressão para os cumulantes e para isso irei recorrer para a função característica. Antes de usar a transformada de Fourier é necessário escrever $p_l$ como uma densidade de probabilidade, 
\begin{equation}
\begin{split}
f(x) = \sum_l p_l\delta(x-l) \quad,\\
f(x) = \sum_l \frac{e^{-\lambda} \lambda^l\delta(x-l)}{l!} \quad.
\end{split}
\end{equation}
Assim a função característica vem naturalmente, 
\begin{equation}
\begin{split}
g(k) = \int e^{ikx}\sum_l \frac{e^{-\lambda} \lambda^l\delta(x-l)}{l!} dx\\
g(k) = \sum_l \frac{e^{-\lambda} \lambda^l e^{ikl}}{l!} \quad.
\end{split}
\end{equation}
A exponencial em $\lambda$ sai para fora da soma e fatorando o expoente $l$, obtemos a expressão:
\begin{equation}
g(k) = e^{-\lambda}\sum_l \frac{(\lambda^l e^{ik})^l}{l!} = e^{-\lambda}e^{\lambda e^{ik}} = e^{\lambda(e^{ik} - 1)} \quad. 
\end{equation}
A função geradora de cumulantes é a expansão do $\ln$ da função característica,
\begin{equation}
\ln g(k) = \sum\limits{n=1} \frac{(ik)^n \kappa_n}{n!},
\end{equation}
e aplicando para a função característica encontrada,
\begin{equation}
\ln g(k) = \lambda(e^{ik} - 1) = \lambda\sum_{n=1} \frac{1}{n!}\frac{d^n e^{ik}}{dk}\Big\rvert_{k=0} = \lambda\sum_{n=1} \frac{(ik)^n}{n!} \quad.
\end{equation}
Comparado a expansão da função característica com a função geradora de cumulantes concluí-se que todos os cumulantes são iguais e assumem o valor de $\lambda$.
\begin{equation}
\kappa_n = \lambda, \quad n=1, 2, 3, \cdots
\end{equation}

\section*{Exercício 3}

\section*{Exercício 7}
A equação de Fokker-Planck pode ser escrita como uma equação de continuidade,
\begin{equation}
\frac{\partial P(x,t)}{\partial t} = \sum_i\left( \frac{\partial J(x, t)}{\partial t}\right),
\end{equation}
onde $J$ é a corrente de probabilidade. 

A entropia é definida como:
\begin{equation}
S(t) = -\int P(x,t) \ln P(x,t) dx
\end{equation}
\end{document}